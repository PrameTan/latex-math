\documentclass[12pt]{book}

\usepackage{xltxtra, amsmath, xfrac,fontspec, amsfonts, mathtools, enumitem, indentfirst} 
\usepackage{amsthm}
\usepackage{tabu}

\setlength{\parindent}{0.5in}
\usepackage{multicol}
\setlength{\columnsep}{1cm}

\usepackage{color}   %May be necessary if you want to color links
\definecolor{maroon}{rgb}{0.5,0,0}
\usepackage{hyperref}
\hypersetup{
	colorlinks=true, %set true if you want colored links
	linktoc=all,     %set to all if you want both sections and subsections linked
	linkcolor=maroon,  %choose some color if you want links to stand out
}

\usepackage{fancyhdr}

\usepackage{subfiles}

\usepackage{imakeidx}
\makeindex[title=Index]
%\usepackage[T1]{fontenc}
%\XeTeXlinebreaklocale "th"
%\XeTeXlinebreakskip = 0pt plus 0pt   %
%\sloppy

%\defaultfontfeatures{Mapping=tex-text} 
%\setmainfont{TeX Gyre Termes}	
%\setsansfont{TeX Gyre Heros}
%\setmonofont{TeX Gyre Cursor}


%\newfontfamily{\thaifont}[Scale=MatchUppercase,Mapping=tex-text]{TH Sarabun New}
%\newenvironment{thailang}
%{\thaifont}
%{}
%\usepackage[Latin,Thai]{ucharclasses}
%\setTransitionTo{Thai}{\begin{thailang}}
%	\setTransitionFrom{Thai}{\end{thailang}}
%\usepackage{setspace}
%\onehalfspacing
%\usepackage{polyglossia}
%\setdefaultlanguage{english}
%\setotherlanguage{thai}
%\newlength{\mylength}

\usepackage[top = 1.2in, bottom = 1in, left=1 in, right = 1in]{geometry}

\DeclareMathOperator\Arg{arg}
\DeclareMathOperator\cis{cis}
\DeclareMathOperator{\adj}{adj}

\newcommand{\dg}{^\circ}
\newcommand{\ii}{\item}
\newcommand{\ol}[1]{\overline{#1}}

\newcommand{\iv}{^{-1}}

\newcommand{\lbr}{\left\lbrace}
\newcommand{\rbr}{\right\rbrace}
\newcommand{\dps}{\displaystyle}

\newcommand{\CC}{\mathbb C}
\newcommand{\FF}{\mathbb F}
\newcommand{\NN}{\mathbb N}
\newcommand{\QQ}{\mathbb Q}
\newcommand{\RR}{\mathbb R}
\newcommand{\ZZ}{\mathbb Z}

\newcommand{\ra}{\rightarrow}
\newcommand{\lra}{\leftrightarrow}
\newcommand{\we}{\wedge}
\newcommand{\ve}{\vee}

\theoremstyle{definition}
\newtheorem{definition}{Definition}[section]
\newtheorem{example}{Example}[section]
\newtheorem{theorem}{Theorem}[chapter]
\newtheorem{lemma}{Lemma}[section]
\newtheorem{corollary}{Corollary}[section]
%\newtheorem{problem}{โจทย์ข้อที่}

\theoremstyle{remark}
%\newtheorem*{remark}{ข้อสังเกต}
\newtheorem*{note}{Note}

\pagestyle{fancy}%

\renewcommand{\chaptermark}[1]{%
	\markboth{#1}{}}
\renewcommand{\sectionmark}[1]{%
	\markright{\thesection\ #1}{}}		

\providecommand\gi{}
% can be useful to refer to this outside \Set
\newcommand\SetSymbol[1][]{%
	\nonscript\:#1\vert
	\allowbreak
	\nonscript\:
	\mathopen{}}
\DeclarePairedDelimiterX\Set[1]\{\}{%
	\renewcommand\gi{\SetSymbol[\delimsize]}
	#1
}
\setlength{\headheight}{18pt}

\AtBeginDocument{	
	\DeclarePairedDelimiter\abs{\lvert}{\rvert}
	\DeclarePairedDelimiter\norm{\lVert}{\rVert}
	\DeclareMathOperator{\tr}{tr}
	
	\DeclareSymbolFont{AMSb}{U}{msb}{m}{n}
	\DeclareSymbolFontAlphabet{\mathbb}{AMSb}
	
	\DeclareSymbolFont{operators}   {OT1}{cmr} {m}{n}
	\DeclareSymbolFont{letters}     {OML}{cmm} {m}{it}
	\DeclareSymbolFont{symbols}     {OMS}{cmsy}{m}{n}
	\DeclareSymbolFont{largesymbols}{OMX}{cmex}{m}{n}
	\SetSymbolFont{operators}{bold}{OT1}{cmr} {bx}{n}
	\SetSymbolFont{letters}  {bold}{OML}{cmm} {b}{it}
	\SetSymbolFont{symbols}  {bold}{OMS}{cmsy}{b}{n}
	\DeclareSymbolFontAlphabet{\mathrm}    {operators}
	\DeclareSymbolFontAlphabet{\mathnormal}{letters}
	\DeclareSymbolFontAlphabet{\mathcal}   {symbols}
	\DeclareMathAlphabet      {\mathbf}{OT1}{cmr}{bx}{n}
	\DeclareMathAlphabet      {\mathsf}{OT1}{cmss}{m}{n}
	\DeclareMathAlphabet      {\mathit}{OT1}{cmr}{m}{it}
	\DeclareMathAlphabet      {\mathtt}{OT1}{cmtt}{m}{n}
}

%\makeatletter
%\renewcommand*\env@matrix[1][*\c@MaxMatrixCols c]{%
%	\hskip -\arraycolsep
%	\let\@ifnextchar\new@ifnextchar
%	\array{#1}}
%\makeatother

\title{\textbf{Mathematics}}
\author{P. Tansuntorn}
\date{Last updated \today}
\begin{document}

	\frontmatter
	\pagestyle{plain}
	\maketitle
	\tableofcontents
	
	\pagestyle{fancy}
		\fancyhf{}%
		\fancyhead[RO,LE]{\thepage}
		\fancyhead[CO]{\textit{\rightmark}}
		\fancyhead[CE]{\textit{\leftmark}}
		\renewcommand{\headrulewidth}{0pt}
		\renewcommand{\headrulewidth}{0pt}
	
	\mainmatter
	\part{Part IA}
	
	\subfile{part-ia/numbers-sets}
	\subfile{part-ia/group}
	\subfile{part-ia/vectors-and-matrices}
	\subfile{part-ia/differential-equations}
	\subfile{part-ia/analysis-i}
	\subfile{part-ia/probability}
	\subfile{part-ia/vector-calculus}
		
	\part{Part IB}
	\chapter{Linear Algebra}
		\section{Vector Spaces}
		
		\section{Linear maps}
		
		\section{Determinant}
		
		\section{Eigenvalues and Eigenvectors}
		
		\section{Duals}
		
		\section{Bilinear Forms}
		
		\section{Inner Product Spaces}
		
		
	\chapter{Groups, Rings and Modules}
		\section{Groups}
		We have gone into details of groups in Part IA.
			\subsection{Basics concepts}
			
			\subsection{Normal subgroups}
			
			\subsection{Sylow subgroups and Sylow theorems}
		\section{Rings}
			\subsection{Definition}
			Rings are abstraction of systems with addition and multiplication. The prototype of rings are the set $\ZZ$ of integers.
			
			We define the general notion of ring in a similar way. We say that a set $R$ with two operations, addition and multiplication, denoted $x + y$ and $x \cdot y$, respectively.
			We write $x \cdot y$ as $xy$ for comprehensiveness.
			\begin{definition}
				A set $R$ is a ring if the following properties are satisfied:
				\begin{enumerate}
					\item $R$ forms an abelian group under addition.
					\item $R$ forms a monoid under multiplication.
					\item The distributive laws hold true, i.e.
					\begin{equation*}
						x(y + z) = xy + xz, (y + z)x = yx + zx.
					\end{equation*}
				\end{enumerate}
			\end{definition}
			\subsection{Ideals}
			
			\subsection{Fields}
			
			\subsection{Factorisation in rings}
			
			\subsection{Rings $\ZZ[a]$ of algebraic integers}
		\section{Modules}
			\subsection{Definition}
			
			\subsection{Submodules}
			
			\subsection{Equivalence of matrices}
			
			\subsection{Finitely generated modules over Euclidean domains}
	\chapter{Analysis II}
		\section{Uniform Convergence}
		
		\section{Uniform Continuity and Integration}
		
		\section{$\RR^n$ as a Normed Space}
		
		\section{Differentiation from $\RR^m$ to $\RR^n$}
		
		\section{Metrice Spaces}
		
		\section{The Contraction Mapping Theorem}
		
	\chapter{Metric and Topological Spaces}
		\section{Metrics}
			\subsection{Definition and examples}
			
			\subsection{Limits and continuity}
			
			\subsection{Open sets and neighbourhoods}
			
			\subsection{Characterising limits and continuity}
			
		\section{Topology}
			\subsection{Metric topologies}
			
		\section{Connectedness}
		
		\section{Compactness}
		
	\chapter{Complex Analysis}
		\section{Analytic Functions}
		
		\section{Contour Integration and Cauchy's Theorem}
		
		\section{Expansions and Singularities}
		
		\section{The Residue Theorem}
		
	\chapter{Complex Methods}
		\section{Analytic Functions}
		
		\section{Contour Integration and Cauchy's Theorem}
		
		\section{Residue Calculus}
		
		\section{Fourier and Laplace Transforms}
	
	\chapter{Geometry}
		
	
	\part{Part II}
	\chapter{Number Theory}
		\section{Basics}
		
		\section{Chinese Remainder Theorem}
		
		\section{Law of Quadratic Reciprocity}
		
		\section{Binary Quadratic Forms}
		
		\section{Distribution of the Primes}
		
		\section{Continued fractions and Pell's equation}
		
		\section{Primality testing}
		
		\section{Factorisation}
	\chapter{Topics in Analysis}
	
	
	\chapter{Coding and Cryptography}
	
	\chapter{Automata and Formal Languages}
		\section{Register machines}
		
		\section{Regular languages and finite-state automata}
		
		\section{Pushdown automata and context-free languages}
	\chapter{Logic and Set Theory}
		\section{Ordinals and Cardinals}
			\subsection{Well-orderings and order-types}
			
		\section{Posets and Zorn's Lemma}
		
		\section{Propositional Logic}
		
		\section{Predicate Logic}
		
		\section{Set Theory}
		
		\section{Consistency}
		
	\chapter{Graph Theory}
		\section{Introduction}
		
		\section{Connectivity and matchinhs}
		
		\section{Extremal graph theory}
		
		\section{Eigenvalue methods}
		
		\section{Graph colouring}
		
		\section{Ramsey theory}
		
		\section{Probabilistic methods}
	\chapter{Galois Theory}
		\section{Fields extensions}
	\chapter{Representation Theory}
		\section{Representations of Finite Groups}
			\subsection{Representations on vector spaces}
		
		\section{Character Theory}
		
		\section{Arithmetic Properties of Characters}
		
		\section{Tensor Products}
		
		\section{Representations of $S^1$ and $SU_2$}
		
		\section{Further Worked Examples}
		
	\chapter{Number Fields}
		\section{Algebraic Number Fields}
		
		\section{Ideals}
		
		\section{Units}
		
		\section{Ideal classes}
		
		\section{Dedekind’s theorem on the factorisation of primes}
	\chapter{Algebraic Topology}
		\section{The Fundamental Group}
		
		\section{Covering Spaces}
		
		\section{The Seifert-Van Kampen Theorem}
		
		\section{Simplicial Complexes}
		
		\section{Homology}
		
		\section{Homology Calculations}
		
	\chapter{Linear Analysis}
	
	\chapter{Analysis of Functions}
	
	\chapter{Riemann Surfaces}
	
	\chapter{Algebraic Geometry}
	
	\chapter{Differential Geometry}
	
	\chapter{Probability and Measure}
	
	\backmatter
		\printindex
%		\begin{equation*}
%			(9)^{2558} = (10 - 1)^{2558} = \binom{2558}{0}10^{2558} - \binom{2558}{1}10^{2557}\cdot 1 + \binom{2558}{2}{10}^{2557}\cdot 1^2 - \cdots - \binom{2558}{2557}5\cdot 1^{2557} + \binom{2558}{2558}1^{2558} \\
\end{document}
		