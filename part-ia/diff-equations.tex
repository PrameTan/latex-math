\documentclass[main.tex]{subfiles}
\begin{document}
	\chapter{Differential Equations}
		\section{Basic Calculus}
			\subsection{Differentiation}
			\begin{definition}
					The derivative of a function $f(x)$ with respect to $x$, is the rate of change of $f(x)$ at $x$, is defined as
					\begin{equation}
						\frac{df}{dx} = \lim_{h \ra 0} \frac{f(x + h) - f(x)}{h}.
					\end{equation}
					The function $f$ is differentiable at $x$ if the limit exists. We may write $\frac{df}{dx} = f'(x)$. And more generally, $\frac{d^n}{dx^n} f(x) = f^{(n)}(x)$ is the $n$-th derivative of $f$.
			\end{definition}
			We shall adopt the convention that $f'(x)$ is the derivative with respect to the argument. For example, $f'(2x) = \frac{df}{d(2x)}$.
			
			\subsection{Big $O$ and small $o$ notation}
				\begin{definition}
					We say that $f(x) = o(g(x))$ as $x \ra x_0$ if $\lim_{x\ra x_0} \frac{f(x)}{g(x)} = 0$. That is, $f(x)$ is much \textit{smaller} than $g(x)$.
				\end{definition}
				\begin{definition}
					We say that $f(x) = O(g(x))$ as $x \ra x_0$ if $\frac{f(x)}{g(x)}$ is bounded as $x \ra x_0$. That is, $f(x)$ is as big as $g(x)$.
				\end{definition}
				The definition of $O$ does not requires that $\lim_{x \ra x_0}\frac{f(x)}{g(x)}$ exists; $\sin x = O(1)$ as $x \ra \infty$ but $\lim_{x \ra \infty} \sin x$ does not exists.
				
				\begin{theorem}
					Let $f$ be a function differentiable at $x_0$, then
					\begin{equation}
						f(x_0 + h) = f(x_0) + f'(x_0)h + o(h)
					\end{equation}
				as $h \ra 0$.
				\end{theorem}
				\begin{proof}
					From the definition of differentiation and $o$,
					\begin{equation}
						f'(x_0) = \frac{f(x_0 + h) - f(x_0)}{h} - \frac{o(h)}{h}.
					\end{equation}
					The result follows.
				\end{proof}		
				\subsection{Rules of differentiation}
			\begin{theorem}[Chain rule]
				Let $f(x) = F(g(x))$, $F$ is differentiable at $g(x)$ and $g$ is differentiable at $x$, then
				\begin{equation*}
					\frac{df}{dx} = \frac{dF}{dg}\frac{dg}{dx}.
				\end{equation*}
			\end{theorem}	
			\begin{proof}
				We have
				\begin{align*}
				\frac{df}{dx} & = \lim_{h \ra 0} \frac{F(g(x + h)) - F(g(x))}{h} \\
							& = \lim_{h \ra 0} \frac{F(g(x) + hg'(x) + o(h)) - F(g(x))}{h}\\
							& = \lim_{h \ra 0} \frac{}{den}
				\end{align*}
			\end{proof}
		\section[1st-order LDEs]{First-order Linear Differential Equations}
			\subsection{Equations with constant coefficients}
				
			\subsection{Equations with non-constant coefficients}
			
		\section{Nonlinear first-order equations}
		
		\section[Higher-order LDEs]{Higher-order Linear Differential Equations}
		
		\section{Multivariate Functions}
			
\end{document}