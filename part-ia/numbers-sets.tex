\documentclass[main.tex]{subfiles}

\begin{document}
	\chapter{Numbers and Sets}
	
	\section{Introduction to number systems and logic}
	
	\section{Sets, relations and functions}
		\subsection{Union, intersection and equality of sets}
		
		\subsection{Indicator functions}
		
		\subsection{Functions}
		
		\subsection{Relations and equivalence relations}
		
		\subsection{The Inclusion-Exclusion Principle}
		
	\section{The integers}
		\subsection{Natural numbers}
		
	\section{Elementary number theory}
		\subsection{Prime numbers}
			\begin{definition}
			 For two integers $a$ and $b$, $a$ \textit{divides} $b$ if there exists an integer $k$ such that $b = ak$. We call $a$ a \textit{factor} of $b$ and write $a \mid b$.
			\end{definition}
			
			\begin{definition}
				A number $p$ is \textit{prime} if its divisors are only 1 and itself. A number which is not prime is called a \textit{composite} number.
			\end{definition}
			\begin{theorem}
				Every number greater than 1 has a prime factor.
			\end{theorem}
			\begin{proof}
				We proceed by induction. Note that 2 obviously has a prime factor 2. Suppose that every number less than $m$ has a prime factor, we need to show that $m$ also has a prime factor. If $m$ is prime then we are done. If $m$ is not prime then there exists $a, b \in\NN$ with $a \leq m$ such that $ab = m$ and $a \neq 1$. Then by the hypothesis, $a$ has a prime factor. That prime factor must also divide $m$. Thus every number greater than 1 has a prime factor.
			\end{proof}
			This proof of infinitude of prime is first described by Euclid.
			\begin{theorem}
				There are infinitely many prime numbers.
			\end{theorem}
			\begin{proof}
				Suppose there are only finitely many prime numbers, denoted $p_1,\ldots, p_k$. Consider the number obtained by multiplying all primes in the list, and then adding one; $p_1p_2\cdots p_k + 1$. This number is obviously greater than 1, and so it must have a prime factor $q$. It then follows that $q$ must be one of the finitely many primes in the list. But for all $p_i$ with $1 \leq i \leq k$, $p_i \nmid p_1p_2\cdots p_k + 1$. This means that $q$ is not equal to any of the prime in the list, a contradiction. 
			\end{proof}
		\subsection{Euclid's algorithm}
		
		\subsection{Solution in integers of $ax + by = c$.}
		
		\subsection{Modular arithmetic}
		
		\subsection{Chinese remainder theorem}
		
		\subsection{Wilson's theorem}
		
	\section{The real numbers}
		\subsection{Least upper bounds}
		
		\subsection{Sequences and series}
		
		\subsection{Irrationality of $\sqrt{2}$ and $e$}
		What does it mean for a number to be rational? Recalls the definition of a rational number, which says that a number $a$ is rational if it can be expressed in the form
		\begin{equation*}
			a = \frac{p}{q}
		\end{equation*}
		for relatively prime integers $p, q$ with $q \neq 0$.
		
		We start by the classic proof of irrationality of $\sqrt{2}$.
		\begin{theorem}
			$\sqrt{2}$ is irrational.
		\end{theorem}
		\begin{proof}
			Suppose $\sqrt{2}$ is rational, and $\sqrt{2} = \frac{p}{q}$ with $(p, q) = 1$ and $q \neq 0$. Then $(\sqrt{2})^2 = 2 = \frac{p^2}{q^2}$, so $2q^2 = p^2$. Therefore $2 \mid p^2$; it follows that $p$ is even. But then $p = 2p_0$ for some integer $p_0$, which means that $q^2 = 2{(p_0)}^2$ and $q$ is even. But this contradicts our assumption that $p$ and $q$ are relatively prime.
		\end{proof}
	
		More generally,
		\begin{theorem}
			$\sqrt{p}$ is irrational if $p$ is prime.
		\end{theorem}
		\begin{proof}
			We provide another proof using unique factorisation of integers.
			Assume that $\sqrt{p}$ is a rational number, and that $\sqrt{p} = \frac{a}{b}$, with coprime $a, b$ and $b \neq 0$. If $b = 1$, then $p$ must divide $a^2$, then it divides $a$, which is absurd. Then there exists a prime $q$ in the factorisation of $b$ such that $q \nmid a$, or else they have a common factor.
			
			Now consider $2 = \frac{a^2}{b^2}$. $a^2$ is factored into the product of primes of $a$, but squared. The prime factor of $b^2$ includes $q^2$. As so the fraction $\frac{a^2}{b^2}$ cannot be reduced to an integer, contradicting $2 = \frac{a^2}{b^2}$.
		\end{proof}
		We can extend the result to the following theorem.
		\begin{theorem}
			$\sqrt{\frac{p}{q}}$ is rational if and only if $p$ and $q$ are perfect squares.
		\end{theorem}
		Even more generally,
		\begin{theorem}
			If an integer $a$ is not an exact $k$-th power of another integer then $\sqrt[k]{a}$ is irrational.
		\end{theorem}
		We now provide a proof that $e$ is irrational, starting with the definition of $e$.
		\begin{definition}
			The number $e$ is defined as
			\begin{equation*}
				 e = \sum_{n = 0}^{^\infty}\frac{1}{n!} = \lim_{n \ra \infty}\left( 1 + \frac{1}{n}\right)^n.
			\end{equation*}
		\end{definition}
		We will show later on that the two definition is indeed equal. The proof of irrationality of $e$ will use the fact that 
		\begin{equation*}
			e = \sum_{n = 0}^{^\infty}\frac{1}{n!} = 1 + \frac{1}{1!} + \frac{1}{2!} + \frac{1}{3!} + \cdots.
		\end{equation*}
		Note that $2 = 1 + 1 < e = 1 + \frac{1}{1!} + \frac{1}{2!} + \frac{1}{3!} + \cdots < 1 + \left( 1 + \frac{1}{2} + \frac{1}{2^2} + \frac{1}{2^3} + \cdots\right) = 3$, that is $e$ is bounded between 2 and 3. Now we present the proof of irrationality of $e$, as presented by Joseph Fourier.
		\begin{theorem}
			$e$ is irrational.
		\end{theorem}
		\begin{proof}
			Suppose $e$ is rational and with usual condition $(a,b) = 1$, $e = \frac{a}{b}$. Define
			\begin{equation}
				x = b!\left(e - \sum_{n = 0}^{b} \frac{1}{n!}\right).
			\end{equation}
			This renders $x$ an integer, for if we substitute $e = \frac{a}{b}$,
			\begin{equation*}
				x = b!\left(\frac{a}{b} - \sum_{n = 0}^{b} \frac{1}{n!}\right) = a(b - 1)! - \sum_{n = 0}^{b} \frac{b!}{n!}.
			\end{equation*}
			For $0 \leq n \leq b$, $n!$ divides entirely into $b!$, and so the sum is an integer.
			
			Notice that we are using an idea that the difference between the fast-converging series expansion of $e$ and $\sum_{n = 0}^{b} \frac{1}{n!}$ multiplied by $b!$ is still less than 1, thus making $x$ an integer between 0 and 1. This would give us a contradiction.
			
			Let's bound $x$ first by showing that it is indeed positive, since
			\begin{equation}
			x = b!\left(e - \sum_{n = 0}^{b} \frac{1}{n!}\right) = b!\left(\sum_{n = 0}^{\infty} \frac{1}{n!} - \sum_{n = 0}^{b} \frac{1}{n!}\right) = \sum_{n = b + 1}^{\infty} \frac{b!}{n!},
			\end{equation}
			and all of its terms is positive, so $x > 0$.
			
			Consider ${b!}/{n!}$. For all term $n \geq b + 1$,
			\begin{equation*}
				\frac{b!}{n!} = \frac{1}{(b + 1)(b + 2)\cdots(b + (n - b))} < \frac{1}{(b + 1)^{n - b}}.
			\end{equation*}
			The inequality is strict for $n > b + 1$, we now have
			\begin{equation}
				x = \sum_{n = b + 1}^{\infty} \frac{b!}{n!} < \sum_{n = b + 1}^{\infty} \frac{1}{(b + 1)^{n - b}} = \sum_{k = 1}^{\infty} \frac{1}{(b + 1)^{k}} = \frac{1}{b + 1}\left(\frac{1}{1 - \frac{1}{b + 1}}\right) = \frac{1}{b} < 1.
			\end{equation}
			A contradiction.
		\end{proof}
		Later in the 19th century, $e$ is proven to be transcendental, i.e. $e$ is not a root of any polynomial with rational coefficient, by Charles Hermite. Furthermore the result of the Lindemann-Weierstrass theorem indicates that $e^a$ is transcendental if $a$ is rational and non-zero. The same theorem also shows that $\pi$ is transcendental.
		
		\subsection{Decimal expansions}
			
		\subsection{Construction of a transcendental number}
		
	\section{Countability and uncountability}
		
	
\end{document}