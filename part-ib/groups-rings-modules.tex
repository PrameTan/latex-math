\documentclass[main.tex]{subfiles}

\begin{document}
	\chapter{Groups, Rings and Modules}
		\section{Groups}
		We have gone into details of groups in Part IA.
			\subsection{Basics concepts}
			
			Let $\theta\colon G \to H$ be a group homomorphism.
			\begin{definition}
				The \textbf{kernel} of $\theta$ is the set $\ker\theta = \Set{g \in G \colon \theta(g) = 1}$ of all elements in $G$ that are mapped to $1 \in H$.
			\end{definition}
			\begin{theorem}
				For any element $x \in \ker\theta$ and $g \in G$, we have $g^{-1} x g \in \ker\theta$.
			\end{theorem}
		
			\subsection{Normal subgroups}
			We begin with the definition of normal subgroups. Recall that a subgroup $H$ of $G$ is a subset of $G$ that form a group under the same operation as $G$.
			
			\begin{definition}
				content...
			\end{definition}
			\subsection{Isomorphism Theorems}
			
			\subsection{Sylow subgroups and Sylow theorems}
		\section{Rings}
			\subsection{Definition}
			Rings are abstraction of systems with addition and multiplication. The prototype of rings are the set $\ZZ$ of integers.
			
			We define the general notion of ring in a similar way. We say that a set $R$ with two operations, addition and multiplication, denoted $x + y$ and $x \cdot y$, respectively.
			\begin{definition}
				A set $R$, together with two binary operations $+$ and $\cdot$, is a \textit{ring} if it satisfies all of the following properties 
				\begin{itemize}
					\item [(A1)] $a, b \in R$ implies $a + b \in \RR$,
					\item [(A2)] $a + b = b + a$ for all $a, b\in R$,
					\item [(A3)] $(a + b) + c = a + (b + c)$ for all $a, b, c \in  R$,
					\item [(A4)] there exists an element, denoted $0$ such that $a + 0 = a$ for all $a \in R$,
					\item [(A5)] for each $a \in R$, there exists some element $b \in R$ such that $a + b = 0$,
					\item [(M1)] $a, b \in R$ implies that $a \cdot b \in R$
					\item [(M2)] $a \cdot (b \cdot c) = (a \cdot b) \cdot c$ for all $a, b, c\in R$
					\item [(D)] $a\cdot(b + c) = a\cdot b + a\dot c$ and $(b + c)\cdot a = b\cdot a + c \cdot a$ for $a, b, c\in R$
				\end{itemize}
			\end{definition}
			We write $x \cdot y$ as $xy$ for comprehensiveness. There is another formulation of the definition of a ring.
			\begin{definition}
				A set $R$ is a ring if the following properties are satisfied:
				\begin{enumerate}
					\item $R$ forms an abelian group under addition.
					\item $R$ forms a monoid under multiplication.
					\item The distributive laws hold true.
				\end{enumerate}
			\end{definition}
			Note that both definition requires multiplication to be associative. Of course, non-associative ring exists, but we shall not confer with them now.
			
			Furthermore, even if our prototype is $\ZZ$, there are few properties of $\ZZ$ missing from the definition. For example, we do not impose that there exists an element $1 \in R$ so that $a \cdot 1 = 1 \cdot a = a$ for every $a \in R$. Such ring is called \textit{ring with unit}.
			
			You might see that elements of a ring do not need to commute under multiplication. But for such special occasion, i.e. $a\cdot b = b \cdot a$ for $a, b\in R$, they are called \textit{commutative ring}.
			
			Lastly, it is not true in general that, if $ab = 0$ then $a = 0$ or $b = 0$. The ring with the above property is called a \textit{domain}.
			\begin{definition}
				A commutative ring $R$ is an \textit{integral domain}\index{integral domain} if $ab = 0$ implies $a = 0$ or $b = 0$.
			\end{definition}
			
			\begin{definition}
				A ring $R$ is called a division ring if for every $a \neq =0$ there is an element $b \in R$ such that $ab = ba = 1$. Later we shall denote $b$ by $a^{-1}$.
			\end{definition}
		
		\begin{definition}
			A ring $R$ is said to be a field if it is a commutative division ring.
		\end{definition}
			Direct verification should show that $\QQ, \RR$ and $\CC$ are fields.
			\subsection{Ideals}
			
			\subsection{Fields}
			
			\subsection{Factorisation in rings}
			
			\subsection{Rings $\ZZ[a]$ of algebraic integers}
			
		\section{Modules}
			\subsection{Definition}
			
			\subsection{Submodules}
			
			\subsection{Equivalence of matrices}
			
			\subsection{Finitely generated modules over Euclidean domains}
\end{document}