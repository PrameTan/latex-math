\documentclass[main.tex]{subfiles}

\begin{document}
	\chapter{Groups, Rings and Modules}
		\section{Groups}
		We have gone into details of groups in Part IA.
			\subsection{Basics concepts}
			
			\subsection{Normal subgroups}
			
			\subsection{Sylow subgroups and Sylow theorems}
		\section{Rings}
			\subsection{Definition}
			Rings are abstraction of systems with addition and multiplication. The prototype of rings are the set $\ZZ$ of integers.
			
			We define the general notion of ring in a similar way. We say that a set $R$ with two operations, addition and multiplication, denoted $x + y$ and $x \cdot y$, respectively.
			We write $x \cdot y$ as $xy$ for comprehensiveness.
			\begin{definition}
				A set $R$ is a ring if the following properties are satisfied:
				\begin{enumerate}
					\item $R$ forms an abelian group under addition.
					\item $R$ forms a monoid under multiplication.
					\item The distributive laws hold true, i.e.
					\begin{equation*}
						x(y + z) = xy + xz, (y + z)x = yx + zx.
					\end{equation*}
				\end{enumerate}
			\end{definition}
			\subsection{Ideals}
			
			\subsection{Fields}
			
			\subsection{Factorisation in rings}
			
			\subsection{Rings $\ZZ[a]$ of algebraic integers}
		\section{Modules}
			\subsection{Definition}
			
			\subsection{Submodules}
			
			\subsection{Equivalence of matrices}
			
			\subsection{Finitely generated modules over Euclidean domains}
\end{document}