\documentclass[main.tex]{subfiles}

\begin{document}
	\chapter{Metric and Topological Spaces}
	\section{Metrics}
	\subsection{Definition and examples}
		We extend the notion of continuity, first to $\RR^2$.
		But let's start from the case of continuity in $\RR$. What does it mean for $f$ to be continuous at $a$? One might start form the definition that
		\begin{equation*}
			\forall\epsilon > 0;\quad \exists\delta > 0. \abs{f(x) - f(a)} < \epsilon \text{ for } \abs{x - a} < \delta.
		\end{equation*}
		That is, we can make the distance between $f(x)$ and $f(a)$ as small as we want, by choosing $x$ so that the distance $\abs{x - a}$ between $x$ and $a$ is sufficiently small. What about $\RR^2$?
		\begin{definition}
			A function $f \colon \RR^n \ra \RR$ is \textit{continous} at a point $a = (a_1, a_2, \ldots, a_n) \in \RR^n$, if given $\epsilon >0$ there exists $\delta > 0$ such that $\abs{f(x) - f(a)} < \epsilon$ for every $x = (x_1, x_2 \ldots, x_n)$ satisfying
			\begin{equation*}
				\sqrt{\sum_{i = 1}^{n} \left(x_i - a_i\right)^2} < \delta.
			\end{equation*}
		\end{definition}
	\subsection{Limits and continuity}
	
	\subsection{Open sets and neighbourhoods}
	
	\subsection{Characterising limits and continuity}
	
	\section{Topology}
	\subsection{Definition}
	
	\subsection{Metric topologies}
	
	\subsection{Neighbourhoods}
	
	\subsection{Hausdorff spaces}
	
	\subsection{Homeomorphisms}
	
	\subsection{Topological and non-topological properties}
	
	\subsection{Completeness}
	
	\subsection{Subspace, quotient and product topologies}
	
	\section{Connectedness}
	\subsection{Definition}
	
	\subsection{Components}
	
	\subsection{Path-connectedness}
	
	\section{Compactness}
	We shall define compactness by first introducing the idea of covers.
	\begin{definition}
		Suppose $X$ is a set and $A \subseteq X$. A family $\Set{U_i \colon i \in I}$ of subsets of $X$ is called a \textit{cover} \index{Cover} for $A$ if
		\begin{equation*}
			A \subseteq\bigcup_{i \in I} U_i.
		\end{equation*}
	\end{definition}
\end{document}