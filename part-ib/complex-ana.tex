\documentclass[main.tex]{subfiles}

\begin{document}
\chapter{Complex Analysis}
	\section{Analytic Functions}
	\subsection{Complex differentiation}
	We first start with the basic definition of limit for complex functions.
	\begin{definition}
		The function $f$ is said to have the limit $A$ as $x$ tends to $a$,
		\begin{equation*}
			\lim_{x\to a} f(x) = A,
		\end{equation*}
		if and only if the following is true:
		
		For every $\epsilon > 0$ there exists a real number $\delta > 0$ with the property that $\abs{f(x) - A} < \epsilon$ for all values of $X$ such that $\abs{x - a} < \delta$ and $x \neq a$.
	\end{definition}
	Note that the definition is the same to those of limit of a real function. This is possible since the absolute function admits both real and complex numbers. The well-known results concerning the limit of a sum, a product and a quotient of limits is preserved.
	
	Note that we also have the following properties:
	\begin{enumerate}
		\item $\lim_{x \to a} \overline{f(x)} = \overline{A}$.
		\item $\lim_{x \to a} \Re f(x) = \Re A$.
		\item $\lim_{x \to a} \Im f(x) = \Im A$.
	\end{enumerate}
	\begin{definition}
		The function $f$ is said to be continuous if $\lim_{x \to a} f(x) = f(a)$.
	\end{definition}
	\begin{definition}
		A complex-valued function $f$ defined on an \textit{open} subset $G$ of $\CC$ is differentiable at $z\in G$ if
		\begin{equation*}
			\lim_{h\to 0} \frac{f(z + h) - f(z)}{h}
		\end{equation*}
		exists. When the limit does exist it is denoted by $f'(z)$.
	\end{definition}

	 Let us consider when $f$ is a real function, that is $f(z)$ is real for all value of $z$. Then the quotient
	 \begin{equation*}
		 \frac{f(z + h) - f(z)}{h}
	 \end{equation*}
	is real if $h$ is real; and if $h = ia$ is purely imaginary, then
	\begin{equation*}
	\frac{f(z + ia) - f(z)}{ia}
	\end{equation*}
	is imaginary. It follows that $f'(z) = 0$ for all $z$ in the domain. Thus a real function of a complex variable must either has the derivative zero, else the derivative does not exist.
	
	\subsection{Conformal mappings}
	
	\section{Contour Integration and Cauchy's Theorem}
		
	\section{Expansions and Singularities}
		
	\section{The Residue Theorem}
				
\end{document}