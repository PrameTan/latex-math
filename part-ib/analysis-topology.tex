\documentclass[main.tex]{subfiles}

\begin{document}
	\chapter{Analysis and Topology}
	\section{Uniform convergence and uniform continuity}
		We first studies the notions of convergence.
		\begin{definition}
			Suppose $(f_n)_{n = 1}^\infty$ is a sequence of real-valued functions, defined on a set $S$. Then $(f_n)_{n = 1}^\infty$ \textit{converges point-wise} to $f$ if for each $s \in S$ and each $\epsilon > 0$ there exists $n_0 \in \NN$ such that $\abs{f_n(s) - f(s)} < \epsilon$ for all $n \geq n_0$.
		\end{definition}
		But point-wise convergence is not adequate. We consider instead uniform convergence, which the number $n_0$ works for all $s$.
		
		\begin{definition}
			We say that the sequence $(f_n)_{n = 1}^\infty$ \textit{converges uniformly} to $f$ on $S$ if for each $\epsilon > 0$ there exists $n_0 \in \NN$ such that $\abs{f_n(s) - f(s)} < \epsilon$ for all $n \geq n_0$ and $s \in S$.
		\end{definition}
		
		
	\section{Metric spaces}
	
	\section{Topological spaces}
	
	\section{Connectedness}
	
	\section{Compactness}
	
	\section{Differentiation from $\RR^m$ to $\RR^n$}
	
\end{document}